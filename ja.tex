\section{Probando JA}

\section{Cuáles operadores unitarios y qué hacen las filas/columnas de la $a$ de qudits}

\begin{align*}\label{eq:a:qutrits}
\left(
\begin{array}{ccccccccc}
 1 & 1 & 1 & 1 & 1 & 1 & 1 & 1 & 1 \\
 1 & 1 & 1 & e^{-\frac{1}{3} (2 i \pi )} & e^{-\frac{1}{3} (2 i \pi )} & e^{-\frac{1}{3} (2 i \pi )} & e^{\frac{2 i \pi }{3}} & e^{\frac{2 i \pi }{3}} & e^{\frac{2 i \pi }{3}} \\
 1 & 1 & 1 & e^{\frac{2 i \pi }{3}} & e^{\frac{2 i \pi }{3}} & e^{\frac{2 i \pi }{3}} & e^{-\frac{1}{3} (2 i \pi )} & e^{-\frac{1}{3} (2 i \pi )} & e^{-\frac{1}{3} (2 i \pi )} \\
 1 & e^{\frac{2 i \pi }{3}} & e^{-\frac{1}{3} (2 i \pi )} & 1 & e^{\frac{2 i \pi }{3}} & e^{-\frac{1}{3} (2 i \pi )} & 1 & e^{\frac{2 i \pi }{3}} & e^{-\frac{1}{3} (2 i \pi )} \\
 1 & e^{\frac{2 i \pi }{3}} & e^{-\frac{1}{3} (2 i \pi )} & e^{-\frac{1}{3} (2 i \pi )} & 1 & e^{\frac{2 i \pi }{3}} & e^{\frac{2 i \pi }{3}} & e^{-\frac{1}{3} (2 i \pi )} & 1 \\
 1 & e^{\frac{2 i \pi }{3}} & e^{-\frac{1}{3} (2 i \pi )} & e^{\frac{2 i \pi }{3}} & e^{-\frac{1}{3} (2 i \pi )} & 1 & e^{-\frac{1}{3} (2 i \pi )} & 1 & e^{\frac{2 i \pi }{3}} \\
 1 & e^{-\frac{1}{3} (2 i \pi )} & e^{\frac{2 i \pi }{3}} & 1 & e^{-\frac{1}{3} (2 i \pi )} & e^{\frac{2 i \pi }{3}} & 1 & e^{-\frac{1}{3} (2 i \pi )} & e^{\frac{2 i \pi }{3}} \\
 1 & e^{-\frac{1}{3} (2 i \pi )} & e^{\frac{2 i \pi }{3}} & e^{-\frac{1}{3} (2 i \pi )} & e^{\frac{2 i \pi }{3}} & 1 & e^{\frac{2 i \pi }{3}} & 1 & e^{-\frac{1}{3} (2 i \pi )} \\
 1 & e^{-\frac{1}{3} (2 i \pi )} & e^{\frac{2 i \pi }{3}} & e^{\frac{2 i \pi }{3}} & 1 & e^{-\frac{1}{3} (2 i \pi )} & e^{-\frac{1}{3} (2 i \pi )} & e^{\frac{2 i \pi }{3}} & 1 \\
\end{array}
\right)
\end{align*}

La acción de un canal caracterizado por una $\tau$ que sea alguna
de las filas o columnas de la matriz \eqref{eq:a:qutrits} 
sobre un estado general de la forma
\begin{align}
\ket{\psi} = \alpha \ket{0} + \beta \ket{1} + \gamma \ket{2}
\end{align}
es intercambiar los estados $\ket{i}$, introducir alguna fase
local o una combinación de ambas. Veamos un ejemplo a continuación.
Consideremos el canal WCE 
\begin{align}
\vec \tau &= \qty(1, 1, 1,
e^{\frac{2}{3}\pi i}, e^{\frac{2}{3}\pi i}, e^{\frac{2}{3}\pi i},
e^{-\frac{2}{3}\pi i}, e^{-\frac{2}{3}\pi i}, e^{-\frac{2}{3}\pi i}) \\
&=\qty(1,1,1,\omega,\omega,\omega,\omega^2,\omega^2,\omega^2),
\label{eq:tau:ejemplo:1}
\end{align}
con $\omega=e^{\frac{2}{3}\pi i}$. Con esta $\tau$ los 
eigenvalores de la matriz de Choi son todos cero, excepto 
uno que es igual a uno. Es decir, el canal asociado al 
vector $\tau$ en \eqref{eq:tau:ejemplo:1} tiene un operador 
de Kraus y, por tanto, es una operación unitaria. 

Como vimos en el ejemplo anterior, los vectores $\vec \tau$,
en general, tienen entradas complejas. Por consiguiente, 
construir las desigualdades $A\vec \tau \geq 0$ debe considerar
dede considerar que las entradas $\tau_{\vec \alpha}$ son 
complejas también.

Dado que $\tau_{\vec \alpha}\in \mathbb{C}$ cambiar de base 
a la matriz $A$ para que $A_{ij}\in \mathbb{R}$ no asegurará 
que $A\vec \tau \in \mathbb{R}$. Recordemos que, del ejemplo
que discutimos en \eqref{eq:tau:ejemplo:1}, las entradas 
del vector $\vec \tau$ sí puede tomar valores complejos.